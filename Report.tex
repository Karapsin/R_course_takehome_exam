% Options for packages loaded elsewhere
\PassOptionsToPackage{unicode}{hyperref}
\PassOptionsToPackage{hyphens}{url}
%
\documentclass[
]{article}
\usepackage{amsmath,amssymb}
\usepackage{iftex}
\ifPDFTeX
  \usepackage[T1]{fontenc}
  \usepackage[utf8]{inputenc}
  \usepackage{textcomp} % provide euro and other symbols
\else % if luatex or xetex
  \usepackage{unicode-math} % this also loads fontspec
  \defaultfontfeatures{Scale=MatchLowercase}
  \defaultfontfeatures[\rmfamily]{Ligatures=TeX,Scale=1}
\fi
\usepackage{lmodern}
\ifPDFTeX\else
  % xetex/luatex font selection
\fi
% Use upquote if available, for straight quotes in verbatim environments
\IfFileExists{upquote.sty}{\usepackage{upquote}}{}
\IfFileExists{microtype.sty}{% use microtype if available
  \usepackage[]{microtype}
  \UseMicrotypeSet[protrusion]{basicmath} % disable protrusion for tt fonts
}{}
\usepackage{xcolor}
\usepackage[left=3cm,right=2.55cm,top=2cm,bottom=2cm]{geometry}
\usepackage{color}
\usepackage{fancyvrb}
\newcommand{\VerbBar}{|}
\newcommand{\VERB}{\Verb[commandchars=\\\{\}]}
\DefineVerbatimEnvironment{Highlighting}{Verbatim}{commandchars=\\\{\}}
% Add ',fontsize=\small' for more characters per line
\usepackage{framed}
\definecolor{shadecolor}{RGB}{248,248,248}
\newenvironment{Shaded}{\begin{snugshade}}{\end{snugshade}}
\newcommand{\AlertTok}[1]{\textcolor[rgb]{0.94,0.16,0.16}{#1}}
\newcommand{\AnnotationTok}[1]{\textcolor[rgb]{0.56,0.35,0.01}{\textbf{\textit{#1}}}}
\newcommand{\AttributeTok}[1]{\textcolor[rgb]{0.13,0.29,0.53}{#1}}
\newcommand{\BaseNTok}[1]{\textcolor[rgb]{0.00,0.00,0.81}{#1}}
\newcommand{\BuiltInTok}[1]{#1}
\newcommand{\CharTok}[1]{\textcolor[rgb]{0.31,0.60,0.02}{#1}}
\newcommand{\CommentTok}[1]{\textcolor[rgb]{0.56,0.35,0.01}{\textit{#1}}}
\newcommand{\CommentVarTok}[1]{\textcolor[rgb]{0.56,0.35,0.01}{\textbf{\textit{#1}}}}
\newcommand{\ConstantTok}[1]{\textcolor[rgb]{0.56,0.35,0.01}{#1}}
\newcommand{\ControlFlowTok}[1]{\textcolor[rgb]{0.13,0.29,0.53}{\textbf{#1}}}
\newcommand{\DataTypeTok}[1]{\textcolor[rgb]{0.13,0.29,0.53}{#1}}
\newcommand{\DecValTok}[1]{\textcolor[rgb]{0.00,0.00,0.81}{#1}}
\newcommand{\DocumentationTok}[1]{\textcolor[rgb]{0.56,0.35,0.01}{\textbf{\textit{#1}}}}
\newcommand{\ErrorTok}[1]{\textcolor[rgb]{0.64,0.00,0.00}{\textbf{#1}}}
\newcommand{\ExtensionTok}[1]{#1}
\newcommand{\FloatTok}[1]{\textcolor[rgb]{0.00,0.00,0.81}{#1}}
\newcommand{\FunctionTok}[1]{\textcolor[rgb]{0.13,0.29,0.53}{\textbf{#1}}}
\newcommand{\ImportTok}[1]{#1}
\newcommand{\InformationTok}[1]{\textcolor[rgb]{0.56,0.35,0.01}{\textbf{\textit{#1}}}}
\newcommand{\KeywordTok}[1]{\textcolor[rgb]{0.13,0.29,0.53}{\textbf{#1}}}
\newcommand{\NormalTok}[1]{#1}
\newcommand{\OperatorTok}[1]{\textcolor[rgb]{0.81,0.36,0.00}{\textbf{#1}}}
\newcommand{\OtherTok}[1]{\textcolor[rgb]{0.56,0.35,0.01}{#1}}
\newcommand{\PreprocessorTok}[1]{\textcolor[rgb]{0.56,0.35,0.01}{\textit{#1}}}
\newcommand{\RegionMarkerTok}[1]{#1}
\newcommand{\SpecialCharTok}[1]{\textcolor[rgb]{0.81,0.36,0.00}{\textbf{#1}}}
\newcommand{\SpecialStringTok}[1]{\textcolor[rgb]{0.31,0.60,0.02}{#1}}
\newcommand{\StringTok}[1]{\textcolor[rgb]{0.31,0.60,0.02}{#1}}
\newcommand{\VariableTok}[1]{\textcolor[rgb]{0.00,0.00,0.00}{#1}}
\newcommand{\VerbatimStringTok}[1]{\textcolor[rgb]{0.31,0.60,0.02}{#1}}
\newcommand{\WarningTok}[1]{\textcolor[rgb]{0.56,0.35,0.01}{\textbf{\textit{#1}}}}
\usepackage{graphicx}
\makeatletter
\newsavebox\pandoc@box
\newcommand*\pandocbounded[1]{% scales image to fit in text height/width
  \sbox\pandoc@box{#1}%
  \Gscale@div\@tempa{\textheight}{\dimexpr\ht\pandoc@box+\dp\pandoc@box\relax}%
  \Gscale@div\@tempb{\linewidth}{\wd\pandoc@box}%
  \ifdim\@tempb\p@<\@tempa\p@\let\@tempa\@tempb\fi% select the smaller of both
  \ifdim\@tempa\p@<\p@\scalebox{\@tempa}{\usebox\pandoc@box}%
  \else\usebox{\pandoc@box}%
  \fi%
}
% Set default figure placement to htbp
\def\fps@figure{htbp}
\makeatother
\setlength{\emergencystretch}{3em} % prevent overfull lines
\providecommand{\tightlist}{%
  \setlength{\itemsep}{0pt}\setlength{\parskip}{0pt}}
\setcounter{secnumdepth}{-\maxdimen} % remove section numbering
\usepackage[backend=biber, style=authoryear]{biblatex}
\usepackage[fontsize=12pt]{scrextend}
\usepackage{indentfirst}
\usepackage{setspace}
\usepackage[singlelinecheck=false]{caption}
\usepackage{float}
\usepackage{sectsty}
\usepackage{booktabs}
\usepackage{caption}
\usepackage{longtable}
\usepackage{array}
\usepackage{multirow}
\usepackage{wrapfig}
\usepackage{float}
\usepackage{colortbl}
\usepackage{pdflscape}
\usepackage{tabu}
\usepackage{threeparttable}
\usepackage{threeparttablex}
\usepackage[normalem]{ulem}
\usepackage{makecell}
\usepackage{xcolor}
\usepackage{bookmark}
\IfFileExists{xurl.sty}{\usepackage{xurl}}{} % add URL line breaks if available
\urlstyle{same}
\hypersetup{
  hidelinks,
  pdfcreator={LaTeX via pandoc}}

\author{}
\date{\vspace{-2.5em}}

\begin{document}

\allsectionsfont{\centering}
\subsectionfont{\raggedright}
\subsubsectionfont{\raggedright}

\pagenumbering{gobble}

\begin{centering}

\vspace{3cm}

\vspace{1cm}

\Large
{NATIONAL RESEARCH UNIVERSITY \\ HIGHER SCHOOL OF ECONOMICS}
\normalsize
\\
{\bf International College of Economics and Finance}
\vspace{1cm}


\Large
\doublespacing
Karapsin Danila\\
Gulomjonov Furkatjon\\
\normalsize
Takehome Exam, Topic 8\\
38.04.01 ECONOMICS\\
Master's Programme {\bf 'Financial Economics'}

\vspace{1 cm}

\normalsize
\singlespacing


\mbox{}
\vfill
\normalsize
Moscow 2024

\end{centering}

\newpage

\pagenumbering{arabic}
\setcounter{page}{2}

\tableofcontents

\newpage

\subsection{a) - data loading}\label{a---data-loading}

This section aims to show how did we load data needed for the project.
The code which does the job is contained in the \emph{data\_load.R} file
and duplicated below for convenience. The idea is simple: firstly, we
save a data frame with all cryptocurrencies in the \emph{crypto\_df}
object, then we are using \emph{walk} function from \emph{purrr} package
to load every needed token, then the loaded data is saved in the loaded
data folder in the rds format.

\begin{Shaded}
\begin{Highlighting}[]
\FunctionTok{rm}\NormalTok{(}\AttributeTok{list =} \FunctionTok{ls}\NormalTok{())}
\FunctionTok{gc}\NormalTok{()}
\FunctionTok{library}\NormalTok{(dplyr)}
\FunctionTok{library}\NormalTok{(crypto2)}
\FunctionTok{library}\NormalTok{(purrr)}

\NormalTok{crypto\_df }\OtherTok{\textless{}{-}} 
  \FunctionTok{crypto\_list}\NormalTok{()}\SpecialCharTok{\%\textgreater{}\%}
    \FunctionTok{filter}\NormalTok{(id }\SpecialCharTok{\textless{}=} \DecValTok{10000}\NormalTok{)}

\FunctionTok{walk}\NormalTok{(}\FunctionTok{unique}\NormalTok{(crypto\_df}\SpecialCharTok{$}\NormalTok{id),}
    \SpecialCharTok{\textasciitilde{}}\NormalTok{ crypto\_df}\SpecialCharTok{\%\textgreater{}\%}
       \FunctionTok{filter}\NormalTok{(id }\SpecialCharTok{==}\NormalTok{ .x)}\SpecialCharTok{\%\textgreater{}\%}
       \FunctionTok{crypto\_history}\NormalTok{(}\AttributeTok{convert =} \StringTok{"USD"}\NormalTok{)}\SpecialCharTok{\%\textgreater{}\%}
       \FunctionTok{saveRDS}\NormalTok{(}\FunctionTok{paste0}\NormalTok{(}\StringTok{"loaded\_data}\SpecialCharTok{\textbackslash{}\textbackslash{}}\StringTok{"}\NormalTok{, .x, }\StringTok{".rds"}\NormalTok{))}
\NormalTok{)}
\end{Highlighting}
\end{Shaded}

\newpage

\subsection{b), c) - data cleaning +
forecasting}\label{b-c---data-cleaning-forecasting}

This section provides an overview of the code from \emph{predictions.R}
file. Firstly, we consolidate each data frame loaded on the previous
step into a single data frame using \emph{map\_dfr} function from
\emph{purrr} package. Then, we select only columns which will be needed
further, keeping only those coins for which we have from 60 to 730
observations. Lastly, we compute log returns and also for every coin we
are computing start of the prediction interval, which is equal to
last\_prediction\_day + 1.

Note that \emph{group\_by(id)} in the code below ensures that filtering
is correct, also because of that last\_prediction\_day + 1 is defined
for every coin separately.

\begin{Shaded}
\begin{Highlighting}[]
\FunctionTok{library}\NormalTok{(magrittr)}
\FunctionTok{library}\NormalTok{(purrr)}
\FunctionTok{library}\NormalTok{(forecast)}
\FunctionTok{library}\NormalTok{(dplyr)}

\CommentTok{\# loaded data concatenation + log\_returns + some helpful vars}
\NormalTok{df }\OtherTok{\textless{}{-}} 
  \FunctionTok{list.files}\NormalTok{(}\StringTok{"loaded\_data"}\NormalTok{)}\SpecialCharTok{\%\textgreater{}\%}
    \FunctionTok{map\_dfr}\NormalTok{(}\SpecialCharTok{\textasciitilde{}}\NormalTok{.x}\SpecialCharTok{\%\textgreater{}\%}
              \FunctionTok{paste0}\NormalTok{(}\StringTok{"loaded\_data}\SpecialCharTok{\textbackslash{}\textbackslash{}}\StringTok{"}\NormalTok{, .)}\SpecialCharTok{\%\textgreater{}\%}
              \FunctionTok{readRDS}\NormalTok{()}
\NormalTok{    )}\SpecialCharTok{\%\textgreater{}\%}
  
  \FunctionTok{select}\NormalTok{(id, time\_open, close)}\SpecialCharTok{\%\textgreater{}\%}
  \FunctionTok{group\_by}\NormalTok{(id)}\SpecialCharTok{\%\textgreater{}\%}
  \FunctionTok{filter}\NormalTok{(time\_open}\SpecialCharTok{\%\textgreater{}\%}
          \FunctionTok{length}\NormalTok{()}\SpecialCharTok{\%\textgreater{}\%}
          \FunctionTok{unique}\NormalTok{()}\SpecialCharTok{\%\textgreater{}\%}
          \FunctionTok{between}\NormalTok{(}\DecValTok{60}\NormalTok{, }\DecValTok{730}\NormalTok{)}
\NormalTok{  )}\SpecialCharTok{\%\textgreater{}\%}
  \FunctionTok{arrange}\NormalTok{(id, time\_open)}\SpecialCharTok{\%\textgreater{}\%}
  \FunctionTok{mutate}\NormalTok{(}\AttributeTok{day\_num =} \FunctionTok{row\_number}\NormalTok{())}\SpecialCharTok{\%\textgreater{}\%}
  \FunctionTok{mutate}\NormalTok{(}\AttributeTok{last\_day\_num =} \FunctionTok{max}\NormalTok{(day\_num))}\SpecialCharTok{\%\textgreater{}\%}
  \FunctionTok{mutate}\NormalTok{(}\AttributeTok{log\_return =}\NormalTok{ close}\SpecialCharTok{\%\textgreater{}\%}
                        \StringTok{\textasciigrave{}}\AttributeTok{/}\StringTok{\textasciigrave{}}\NormalTok{(}\FunctionTok{lag}\NormalTok{(close, }\DecValTok{1}\NormalTok{))}\SpecialCharTok{\%\textgreater{}\%} 
                        \FunctionTok{log}\NormalTok{(),}
         \AttributeTok{last\_prediction\_day =}\NormalTok{ last\_day\_num}\SpecialCharTok{/}\DecValTok{2}
\NormalTok{  )}\SpecialCharTok{\%\textgreater{}\%}
  \FunctionTok{filter}\NormalTok{(}\SpecialCharTok{!}\FunctionTok{is.na}\NormalTok{(log\_return))}\SpecialCharTok{\%\textgreater{}\%}
  \FunctionTok{ungroup}\NormalTok{()}\SpecialCharTok{\%\textgreater{}\%}
  \FunctionTok{select}\NormalTok{(}\SpecialCharTok{{-}}\NormalTok{close)}
\end{Highlighting}
\end{Shaded}

Here we are using expanding window approach to forecast log returns. The
idea is simple, firstly, we start a \emph{while} block. Inside that
block for every iteration we are checking how many rows we still need to
fill, if that number is equal to 0, then the job is done and we are
breaking the \emph{while} loop. Otherwise we continue. Then we use
\emph{filter} on df object, to keep only training period. In the end, we
use \emph{group\_by(id)} to compute \emph{auto.arima()} one step ahead
forecast, from which we are taking a middle point. After that, made
prediction are added to \emph{predictions\_df} object and we are
increasing last\_prediction\_day by 1 for those coins for which made
predictions on the current step.

\begin{Shaded}
\begin{Highlighting}[]
\CommentTok{\# expanding window, auto.arima on every step}
\NormalTok{predictions\_df }\OtherTok{\textless{}{-}} \FunctionTok{data.frame}\NormalTok{(}\AttributeTok{id =} \FunctionTok{numeric}\NormalTok{(), }
                             \AttributeTok{day\_num =} \FunctionTok{numeric}\NormalTok{(), }
                             \AttributeTok{predicted\_log\_return =} \FunctionTok{numeric}\NormalTok{()}
\NormalTok{                  )}

\ControlFlowTok{while}\NormalTok{(}\ConstantTok{TRUE}\NormalTok{)\{}
  
\NormalTok{    rows\_left }\OtherTok{\textless{}{-}}\NormalTok{ df}\SpecialCharTok{\%\textgreater{}\%}\FunctionTok{filter}\NormalTok{(day\_num }\SpecialCharTok{\textgreater{}}\NormalTok{ last\_prediction\_day)}\SpecialCharTok{\%\textgreater{}\%}\FunctionTok{nrow}\NormalTok{()}
    \ControlFlowTok{if}\NormalTok{(rows\_left }\SpecialCharTok{==} \DecValTok{0}\NormalTok{)\{}\ControlFlowTok{break}\NormalTok{\}}
    \FunctionTok{print}\NormalTok{(}\FunctionTok{paste0}\NormalTok{(}\StringTok{"rows left: "}\NormalTok{, rows\_left))}
      
\NormalTok{    current\_predictions }\OtherTok{\textless{}{-}} 
\NormalTok{      df}\SpecialCharTok{\%\textgreater{}\%}
        \FunctionTok{filter}\NormalTok{(day\_num }\SpecialCharTok{\textless{}}\NormalTok{ last\_prediction\_day }\SpecialCharTok{+} \DecValTok{1}\NormalTok{)}\SpecialCharTok{\%\textgreater{}\%}
        \FunctionTok{group\_by}\NormalTok{(id)}\SpecialCharTok{\%\textgreater{}\%}
        \FunctionTok{summarise}\NormalTok{(}\AttributeTok{day\_num =} \FunctionTok{max}\NormalTok{(day\_num) }\SpecialCharTok{+} \DecValTok{1}\NormalTok{,}
                  \AttributeTok{predicted\_log\_return =}\NormalTok{ log\_return}\SpecialCharTok{\%\textgreater{}\%}
                                          \FunctionTok{auto.arima}\NormalTok{()}\SpecialCharTok{\%\textgreater{}\%}
                                          \FunctionTok{forecast}\NormalTok{(}\AttributeTok{h =} \DecValTok{1}\NormalTok{)}\SpecialCharTok{\%$\%}
\NormalTok{                                          mean}
\NormalTok{        )}\SpecialCharTok{\%\textgreater{}\%}
        \FunctionTok{ungroup}\NormalTok{()}
      
\NormalTok{      predictions\_df}\SpecialCharTok{\%\textless{}\textgreater{}\%}
        \FunctionTok{rbind}\NormalTok{(current\_predictions)}
    
\NormalTok{    df}\SpecialCharTok{\%\textless{}\textgreater{}\%}
      \FunctionTok{mutate}\NormalTok{(}\AttributeTok{last\_prediction\_day =} \FunctionTok{ifelse}\NormalTok{(id }\SpecialCharTok{\%in\%}\NormalTok{ current\_predictions}\SpecialCharTok{$}\NormalTok{id, }
\NormalTok{                                          last\_prediction\_day }\SpecialCharTok{+} \DecValTok{1}\NormalTok{, }
\NormalTok{                                          last\_prediction\_day)}
\NormalTok{      )}
\NormalTok{\}}
\end{Highlighting}
\end{Shaded}

After previous step is finished, we are adding predictions to the
initial data frame and saving it.

\begin{Shaded}
\begin{Highlighting}[]
\CommentTok{\# add predicted values and save df for further usage}
\NormalTok{df}\SpecialCharTok{\%\textless{}\textgreater{}\%}
  \FunctionTok{left\_join}\NormalTok{(predictions\_df, }\AttributeTok{by =} \FunctionTok{c}\NormalTok{(}\StringTok{"day\_num"}\OtherTok{=}\StringTok{"day\_num"}\NormalTok{, }\StringTok{"id"}\OtherTok{=}\StringTok{"id"}\NormalTok{))}

\FunctionTok{saveRDS}\NormalTok{(df, }\StringTok{"data\_with\_predictions.rds"}\NormalTok{)}
\end{Highlighting}
\end{Shaded}

\newpage

\subsection{d) - prediction intervals}\label{d---prediction-intervals}

The code below is from the \emph{predictions.R} file and it computes all
the required prediction intervals, except the ``FACI'' intervals
mentioned in the task, for which there is no function in the package.
The data frame with computed prediction intervals is then saved in the
rds format. Note again that we use \emph{group\_by(id)} to compute all
the intervals for each coin separately.

\begin{Shaded}
\begin{Highlighting}[]
\FunctionTok{rm}\NormalTok{(}\AttributeTok{list =} \FunctionTok{ls}\NormalTok{())}
\FunctionTok{library}\NormalTok{(magrittr)}
\FunctionTok{library}\NormalTok{(purrr)}
\FunctionTok{library}\NormalTok{(forecast)}
\FunctionTok{library}\NormalTok{(dplyr)}
\FunctionTok{library}\NormalTok{(AdaptiveConformal)}

\NormalTok{df }\OtherTok{\textless{}{-}} \FunctionTok{readRDS}\NormalTok{(}\StringTok{"data\_with\_predictions.rds"}\NormalTok{)}

\NormalTok{intervals\_df }\OtherTok{\textless{}{-}} 
\NormalTok{df}\SpecialCharTok{\%\textgreater{}\%}
  \FunctionTok{filter}\NormalTok{(}\SpecialCharTok{!}\FunctionTok{is.na}\NormalTok{(predicted\_log\_return))}\SpecialCharTok{\%\textgreater{}\%}
  \FunctionTok{group\_by}\NormalTok{(id)}\SpecialCharTok{\%\textgreater{}\%}
  \FunctionTok{summarise}\NormalTok{(}
            \DocumentationTok{\#\#\#\#\#\#\#\#\#\#\#\#\#\#\#\#\#\#\#\#\#\#\#\#\#\#\#\#\#\#\#\#\#\#\#\#\#\#\#\#\#\#\#\#\#\#\#}
            \CommentTok{\# alpha 5\%}
            \AttributeTok{intervals\_AgACI5 =} 
            \FunctionTok{aci}\NormalTok{(}\AttributeTok{method =} \StringTok{"AgACI"}\NormalTok{)}\SpecialCharTok{\%\textgreater{}\%}
            \FunctionTok{update}\NormalTok{(}\AttributeTok{newY =}\NormalTok{ log\_return, }
                   \AttributeTok{newpredictions =}\NormalTok{ predicted\_log\_return}
\NormalTok{            )}\SpecialCharTok{\%$\%}
\NormalTok{            intervals,}

            \CommentTok{\# no FACI in the package :(}
            \AttributeTok{intervals\_dtACI5 =} 
            \FunctionTok{aci}\NormalTok{(}\AttributeTok{method =} \StringTok{"DtACI"}\NormalTok{)}\SpecialCharTok{\%\textgreater{}\%}
            \FunctionTok{update}\NormalTok{(}\AttributeTok{newY =}\NormalTok{ log\_return, }
                   \AttributeTok{newpredictions =}\NormalTok{ predicted\_log\_return}
\NormalTok{            )}\SpecialCharTok{\%$\%}
\NormalTok{            intervals,}
            
            \AttributeTok{intervals\_SF\_OGD5 =} 
            \FunctionTok{aci}\NormalTok{(}\AttributeTok{method =} \StringTok{"SF{-}OGD"}\NormalTok{)}\SpecialCharTok{\%\textgreater{}\%}
            \FunctionTok{update}\NormalTok{(}\AttributeTok{newY =}\NormalTok{ log\_return, }
                   \AttributeTok{newpredictions =}\NormalTok{ predicted\_log\_return,}
                   \AttributeTok{parameters=}\FunctionTok{list}\NormalTok{(}
                   \AttributeTok{gamma =} \FunctionTok{max}\NormalTok{(}\FunctionTok{abs}\NormalTok{(log\_return}\SpecialCharTok{{-}}\NormalTok{predicted\_log\_return))}\SpecialCharTok{/}\FunctionTok{sqrt}\NormalTok{(}\DecValTok{3}\NormalTok{)}
\NormalTok{                     )}
\NormalTok{            )}\SpecialCharTok{\%$\%}
\NormalTok{            intervals,}
            
            \AttributeTok{intervals\_SAOCP5 =} 
            \FunctionTok{aci}\NormalTok{(}\AttributeTok{method =} \StringTok{"SAOCP"}\NormalTok{)}\SpecialCharTok{\%\textgreater{}\%}
            \FunctionTok{update}\NormalTok{(}\AttributeTok{newY =}\NormalTok{ log\_return, }
                   \AttributeTok{newpredictions =}\NormalTok{ predicted\_log\_return,}
                   \AttributeTok{parameters=}\FunctionTok{list}\NormalTok{(}
                   \AttributeTok{D =} \FunctionTok{max}\NormalTok{(}\FunctionTok{abs}\NormalTok{(log\_return}\SpecialCharTok{{-}}\NormalTok{predicted\_log\_return))}\SpecialCharTok{/}\FunctionTok{sqrt}\NormalTok{(}\DecValTok{3}\NormalTok{)}
\NormalTok{                     )}
\NormalTok{                   )}\SpecialCharTok{\%$\%}
\NormalTok{            intervals,}
            
            \DocumentationTok{\#\#\#\#\#\#\#\#\#\#\#\#\#\#\#\#\#\#\#\#\#\#\#\#\#\#\#\#\#\#\#\#\#\#\#\#\#\#\#\#\#\#\#\#\#\#\#\#\#\#}
            \CommentTok{\# alpha 1\%}
            \AttributeTok{intervals\_AgACI1 =} 
            \FunctionTok{aci}\NormalTok{(}\AttributeTok{method =} \StringTok{"AgACI"}\NormalTok{, }\AttributeTok{alpha =} \FloatTok{0.99}\NormalTok{)}\SpecialCharTok{\%\textgreater{}\%}
            \FunctionTok{update}\NormalTok{(}\AttributeTok{newY =}\NormalTok{ log\_return, }
                   \AttributeTok{newpredictions =}\NormalTok{ predicted\_log\_return}
\NormalTok{            )}\SpecialCharTok{\%$\%}
\NormalTok{            intervals,}
            
            \CommentTok{\# no FACI in the package :(}
            \AttributeTok{intervals\_dtACI1 =} 
            \FunctionTok{aci}\NormalTok{(}\AttributeTok{method =} \StringTok{"DtACI"}\NormalTok{, }\AttributeTok{alpha =} \FloatTok{0.99}\NormalTok{)}\SpecialCharTok{\%\textgreater{}\%}
            \FunctionTok{update}\NormalTok{(}\AttributeTok{newY =}\NormalTok{ log\_return, }
                   \AttributeTok{newpredictions =}\NormalTok{ predicted\_log\_return}
\NormalTok{            )}\SpecialCharTok{\%$\%}
\NormalTok{            intervals,}
                                  
            \AttributeTok{intervals\_SF\_OGD1 =} 
            \FunctionTok{aci}\NormalTok{(}\AttributeTok{method =} \StringTok{"SF{-}OGD"}\NormalTok{, }\AttributeTok{alpha =} \FloatTok{0.99}\NormalTok{)}\SpecialCharTok{\%\textgreater{}\%}
            \FunctionTok{update}\NormalTok{(}\AttributeTok{newY =}\NormalTok{ log\_return, }
                   \AttributeTok{newpredictions =}\NormalTok{ predicted\_log\_return,}
                   \AttributeTok{parameters=}\FunctionTok{list}\NormalTok{(}
                   \AttributeTok{gamma =} \FunctionTok{max}\NormalTok{(}\FunctionTok{abs}\NormalTok{(log\_return}\SpecialCharTok{{-}}\NormalTok{predicted\_log\_return))}\SpecialCharTok{/}\FunctionTok{sqrt}\NormalTok{(}\DecValTok{3}\NormalTok{)}
\NormalTok{                     )}
\NormalTok{            )}\SpecialCharTok{\%$\%}
\NormalTok{            intervals,}
            
            \AttributeTok{intervals\_SAOCP1 =} 
            \FunctionTok{aci}\NormalTok{(}\AttributeTok{method =} \StringTok{"SAOCP"}\NormalTok{, }\AttributeTok{alpha =} \FloatTok{0.99}\NormalTok{)}\SpecialCharTok{\%\textgreater{}\%}
            \FunctionTok{update}\NormalTok{(}\AttributeTok{newY =}\NormalTok{ log\_return, }
                   \AttributeTok{newpredictions =}\NormalTok{ predicted\_log\_return,}
                   \AttributeTok{parameters=}\FunctionTok{list}\NormalTok{(}
                   \AttributeTok{D =} \FunctionTok{max}\NormalTok{(}\FunctionTok{abs}\NormalTok{(log\_return}\SpecialCharTok{{-}}\NormalTok{predicted\_log\_return))}\SpecialCharTok{/}\FunctionTok{sqrt}\NormalTok{(}\DecValTok{3}\NormalTok{)}
\NormalTok{                     )}
\NormalTok{            )}\SpecialCharTok{\%$\%}
\NormalTok{            intervals}
\NormalTok{  )}
  
\CommentTok{\# we can do that since the order of days is preserved}
\NormalTok{intervals\_df}\SpecialCharTok{$}\NormalTok{day\_num }\OtherTok{\textless{}{-}} 
\NormalTok{  df}\SpecialCharTok{\%\textgreater{}\%}
  \FunctionTok{filter}\NormalTok{(}\SpecialCharTok{!}\FunctionTok{is.na}\NormalTok{(predicted\_log\_return))}\SpecialCharTok{\%$\%}
\NormalTok{  day\_num}

\NormalTok{intervals\_df}\SpecialCharTok{$}\NormalTok{log\_return }\OtherTok{\textless{}{-}} 
\NormalTok{  df}\SpecialCharTok{\%\textgreater{}\%}
  \FunctionTok{filter}\NormalTok{(}\SpecialCharTok{!}\FunctionTok{is.na}\NormalTok{(predicted\_log\_return))}\SpecialCharTok{\%$\%}
\NormalTok{  log\_return}

\FunctionTok{saveRDS}\NormalTok{(intervals\_df, }\StringTok{"VaR\_data.rds"}\NormalTok{)}
\end{Highlighting}
\end{Shaded}

\newpage

\subsection{e) - backtesting}\label{e---backtesting}

The code in \emph{uc\_cc\_report.R} file allows to create a table which
provided below. The code itself is not provided in that report because
of its ugliness (but it is availiable in the mentioned file). ``+''
indicates that both tests allowed to reject H0, ``uc'' or ``cc''
inidicate that we were able to reject the null only with one test, and,
finally, ``-'' means that we can not reject the null with both tests.

\begingroup\fontsize{8}{10}\selectfont

\begin{longtable}[t]{rllllllll}
\toprule
id & 0.5\_AgACI & 2.5\_AgACI & 0.5\_DtACI & 2.5\_DtACI & 0.5\_SF-OGD & 2.5\_SF-OGD & 0.5\_SAOCP & 2.5\_SAOCP\\
\midrule
22 & cc & cc & cc & cc & + & + & - & -\\
26 & NaN & NaN & NaN & NaN & + & + & NaN & -\\
1612 & NaN & + & NaN & + & + & - & NaN & -\\
1956 & - & - & - & - & + & + & NaN & -\\
2997 & - & cc & - & cc & + & cc & NaN & +\\
\addlinespace
3415 & - & - & - & - & + & + & - & -\\
3906 & - & uc & - & + & + & uc & - & -\\
4100 & uc & - & uc & - & + & + & - & -\\
4474 & cc & cc & - & - & + & + & - & -\\
4513 & cc & - & cc & - & + & - & - & -\\
\addlinespace
4596 & - & cc & - & cc & + & - & - & +\\
4790 & - & - & - & - & NaN & - & NaN & NaN\\
4883 & - & + & - & uc & - & uc & NaN & +\\
5066 & - & - & - & - & + & + & - & -\\
5145 & + & - & + & - & + & + & - & -\\
\addlinespace
5602 & - & - & - & - & + & - & NaN & -\\
6224 & - & - & - & - & + & - & - & -\\
6235 & NaN & + & NaN & + & + & + & - & +\\
6252 & NaN & NaN & NaN & NaN & + & + & NaN & NaN\\
6377 & cc & cc & cc & cc & + & - & - & uc\\
\addlinespace
6759 & - & - & - & - & + & + & - & +\\
6780 & NaN & + & NaN & + & + & + & - & +\\
6827 & cc & cc & cc & cc & + & + & - & +\\
6845 & - & - & - & uc & + & - & - & +\\
6961 & NaN & + & NaN & + & + & cc & NaN & +\\
\addlinespace
7719 & NaN & NaN & NaN & NaN & + & + & NaN & -\\
7760 & + & cc & + & cc & NaN & - & NaN & NaN\\
7961 & - & cc & - & cc & + & cc & - & -\\
7962 & - & - & - & - & - & - & NaN & uc\\
7963 & cc & - & cc & cc & + & cc & - & -\\
\addlinespace
8067 & NaN & - & NaN & - & + & + & - & -\\
8108 & - & - & - & uc & + & + & - & uc\\
8209 & - & - & - & - & cc & cc & NaN & -\\
8266 & - & + & NaN & uc & + & + & - & +\\
8370 & + & cc & + & cc & + & - & - & +\\
\addlinespace
8573 & NaN & NaN & NaN & NaN & NaN & NaN & NaN & NaN\\
8624 & - & + & - & uc & + & - & - & +\\
8724 & NaN & NaN & NaN & NaN & + & + & NaN & -\\
8977 & - & uc & - & - & + & + & - & -\\
9052 & - & uc & - & + & + & cc & - & +\\
\addlinespace
9063 & - & - & - & - & - & - & NaN & -\\
9069 & NaN & + & NaN & + & + & + & - & +\\
9184 & - & - & - & - & - & - & NaN & -\\
9261 & - & - & - & - & + & + & NaN & -\\
9404 & - & - & - & - & + & + & NaN & -\\
\addlinespace
9427 & - & - & - & uc & + & - & - & -\\
9486 & - & + & - & + & cc & - & - & +\\
9531 & - & cc & - & - & + & cc & NaN & +\\
9785 & - & + & NaN & + & + & cc & - & +\\
9808 & NaN & NaN & NaN & + & + & + & NaN & -\\
\addlinespace
9900 & NaN & - & NaN & NaN & + & + & NaN & -\\
\bottomrule
\end{longtable}
\endgroup{}

Note that some cells contain ``NaN''. Sometimes, VaRTest is not working
on our data. We investigated source code from the github of the rugarch
package and found that this happens when we have no log returns which
are below computed VaR (or when we have to few of them). Obviously, in
such a case we can not test anything because lack of observations. So we
have just ``NaN'' in that case.

\newpage

\subsection{f) - saving the results}\label{f---saving-the-results}

For some reason we need to save the results in a very specific format.
The code below does exactly that.

\begin{Shaded}
\begin{Highlighting}[]
\FunctionTok{library}\NormalTok{(dplyr)}
\FunctionTok{library}\NormalTok{(purrr)}

\NormalTok{final\_df }\OtherTok{\textless{}{-}} 
\FunctionTok{readRDS}\NormalTok{(}\StringTok{"data\_with\_predictions.rds"}\NormalTok{)}\SpecialCharTok{\%\textgreater{}\%}
  \FunctionTok{select}\NormalTok{(id, time\_open, day\_num, log\_return, predicted\_log\_return)}\SpecialCharTok{\%\textgreater{}\%}
  \FunctionTok{left\_join}\NormalTok{(}\FunctionTok{readRDS}\NormalTok{(}\StringTok{"VaR\_data.rds"}\NormalTok{)}\SpecialCharTok{\%\textgreater{}\%}
              \FunctionTok{select}\NormalTok{(}\SpecialCharTok{{-}}\NormalTok{log\_return),}
            \AttributeTok{by =} \FunctionTok{c}\NormalTok{(}\StringTok{"id"} \OtherTok{=} \StringTok{"id"}\NormalTok{, }\StringTok{"day\_num"} \OtherTok{=} \StringTok{"day\_num"}\NormalTok{)}
\NormalTok{  )}

\NormalTok{final\_df}\SpecialCharTok{$}\NormalTok{id}\SpecialCharTok{\%\textgreater{}\%}
  \FunctionTok{unique}\NormalTok{()}\SpecialCharTok{\%\textgreater{}\%}
  \FunctionTok{map}\NormalTok{(}\SpecialCharTok{\textasciitilde{}}\NormalTok{final\_df}\SpecialCharTok{\%\textgreater{}\%}
        \FunctionTok{filter}\NormalTok{(id }\SpecialCharTok{==}\NormalTok{ .x)}
\NormalTok{  )}\SpecialCharTok{\%\textgreater{}\%}
  \FunctionTok{saveRDS}\NormalTok{(}\StringTok{"final\_result.rds"}\NormalTok{)}
\end{Highlighting}
\end{Shaded}


\end{document}
